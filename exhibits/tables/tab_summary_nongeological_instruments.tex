\begin{table}[htbp]\centering
\def\sym#1{\ifmmode^{#1}\else\(^{#1}\)\fi}
\caption{Summary statistics for non-geological instruments used for production function estimation}
\begin{tabular*}{1.0\hsize}{@{\hskip\tabcolsep\extracolsep\fill}l*{1}{cccccc}}
\toprule
                                    &\multicolumn{6}{c}{}                                                         \\
                                    &        Mean&    Std. dev&        25th&      Median&        75th&Farmer-crops\\
&\multicolumn{1}{c}{(1)}&\multicolumn{1}{c}{(2)}&\multicolumn{1}{c}{(3)}&\multicolumn{1}{c}{(4)}&\multicolumn{1}{c}{(5)}&\multicolumn{1}{c}{(6)}\\
\midrule
\emph{Land instruments}             &            &            &            &            &            &            \\
Size of the largest parcel (Ha)     &        1.14&        1.73&        0.38&        0.73&        1.36&        8711\\
Size of the 2nd largest parcel (Ha) &        0.14&        0.38&           0&           0&       0.081&        8711\\
Size of the 3rd largest parcel (Ha) &       0.030&        0.16&           0&           0&           0&        8711\\
\emph{Labor instruments}            &            &            &            &            &            &            \\
Adult males                         &        1.66&        1.42&           1&           1&           2&        8711\\
\emph{Capital instruments}          &            &            &            &            &            &            \\
Seed price ('00 INR/kg)             &        1.64&        1.97&        0.31&        0.58&        2.45&        8711\\
\bottomrule
\multicolumn{7}{l}{\footnotesize This table provides summary statistics on the instruments used to generate exogenous variation in productive inputs for production function estimation. All observations are at the farmer-crop level. The first block of summarize land instruments, which consist of the size of the three largest plots of land owned by a farmer. The second block summarizes the main instrument for labor, which is the number of adult males in the household. Finally, the last block summarizes seed prices which affect capital inputs exogenously, assuming the farmet has limited market power. Seed prices for each farmer-crop observation is calculated as the median price of all seed inputs in the feeder in which the farmer is located. Geological instruments are excluded from this summary since they are numerous and heterogenous, and their units are not always easy to interpret.}\\
\end{tabular*}
\end{table}
