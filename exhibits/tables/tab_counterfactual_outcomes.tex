\begin{table}[!ht]
	\centering
		\caption{Counterfactual Production and Social Surplus\label{tab:cfOutcomes}}
\begin{tabular}{lrrrr}
		\toprule
& \multicolumn{2}{c}{Rationing} & \multicolumn{2}{c}{Pricing} \\
\cmidrule(lr){2-3} \cmidrule(lr){4-5}               &Status quo&Optimal&Private cost&Pigouvian\\
&\multicolumn{1}{c}{(1)}&\multicolumn{1}{c}{(2)}&\multicolumn{1}{c}{(3)}&\multicolumn{1}{c}{(4)}\\
		\midrule
		 \addlinespace 
\multicolumn{5}{c}{\emph{A. Profits and social surplus}}\\
		 \addlinespace 
		Profit (INR 000s)&18.72&16.64&19.39&12.90\\
		~~~~$-$ Unpriced power cost (INR 000s)&5.35&4.21&0.00&-4.64\\
		~~~~$-$ Water cost (INR 000s)&5.33&4.18&9.24&4.94\\
\cmidrule(lr){1-1}		Surplus (INR 000s)&8.04&8.25&10.15&12.60\\
		 \addlinespace 
\multicolumn{5}{c}{\emph{B. Input use}}\\
		 \addlinespace 
		      Land (Ha)&0.69&0.69&0.69&0.69\\
		Labor (person-days)&54.81&54.81&54.81&54.81\\
		Capital (INR 000s)&16.58&15.54&20.79&17.81\\
		Water (liter 000s)&1590.65&1248.65&2758.34&1475.47\\
		~~~~Power (kWh per season)&1010.20&793.95&1517.26&768.16\\
		~~~~Hours of use (per day)&5.95&4.67&10.63&5.84\\
		 \addlinespace 
\multicolumn{5}{c}{\emph{C. Output and productivity}}\\
		 \addlinespace 
		Output (INR 000s)&52.78&49.46&66.15&56.68\\
		~~~~Gain in output from status quo (pp)&&  -6&  25&   7\\
		~~~~Gain in output due to input use (pp)&&  -6&  19&   1\\
		~~~~Gain in output due to productivity (pp)&&  -0&   7&   7\\
		$\Cov(\Omega_{Eit},W_{it}^{\alpha_W})$&-0.04&-0.04&0.24&0.25\\
		\bottomrule
\multicolumn{5}{p{0.85\hsize}}{\footnotesize The table shows the outcomes of counterfactual policy regimes with respect to farmer profit, external costs and social surplus. The columns show different policy regimes: the status quo rationing regime, with a ration of 6 hours and a price of INR 0.90 per kWh, a private cost regime, where power is priced at its private marginal cost of INR 6.2 per kWh, and a Pigouvian regime where power is priced at marginal social cost. The rows show the outcome variables in each regime. All outcome variables, except where noted, are mean values at the farmer-by-crop level, where the average farmer plants 2.3 crops. Panel C shows output and the change in output, in percentage points, relative to the status quo value under rationing. Row 3 gives the change in output that would have been achieved from a proportional change in input use for all farmers, equal to the aggregate proportional change in input use in each scenario relative to column 1. Row 4 then gives the residual change in output due to increases in aggregate productivity from the input reallocation. Finally, row 5 reports the covariance between $\Omega_{Eit}$ and the contribution of water input to production.} \\
	\end{tabular}
\end{table}
