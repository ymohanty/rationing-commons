\begin{table}[htbp]\centering
\def\sym#1{\ifmmode^{#1}\else\(^{#1}\)\fi}
\caption{Hedonic regressions of yield on well depth\label{tab:ivYieldDepth}}

\begin{tabular*}{1\hsize}{@{\hskip\tabcolsep\extracolsep\fill}l*{4}{D{.}{.}{-1}}}
	\toprule
	&\multicolumn{1}{c}{OLS}&\multicolumn{1}{c}{OLS}&\multicolumn{1}{c}{IV-PDS}&\multicolumn{1}{c}{IV-PDS}\\
	&\multicolumn{1}{c}{(1)}&\multicolumn{1}{c}{(2)}&\multicolumn{1}{c}{(3)}&\multicolumn{1}{c}{(4)}\\
	\midrule
	
	\addlinespace
	\multicolumn{5}{c}{\emph{Panel A. Yield (quintals per Ha)}} \\
	\addlinespace
	Well depth (1 sd = 187 feet)&       -7.99\sym{***}&       -2.15         &       -6.86\sym{***}&       -3.83         \\
                    &      (1.12)         &      (1.40)         &      (2.66)         &      (2.83)         \\
Toposequence        &                     &         Yes         &         Yes         &         Yes         \\
Soil quality controls&                     &         Yes         &         Yes         &         Yes         \\
Subdivisional effects&                     &         Yes         &         Yes         &         Yes         \\
Plot size effects   &                     &         Yes         &         Yes         &         Yes         \\
                    &                     &                     &                     &                     \\
Mean dep. var       &        46.3         &        46.3         &        46.3         &        46.3         \\
Candidate Instruments&                     &                     &         419         &        1728         \\
Instruments Selected&                     &                     &          15         &          16         \\
Unique Farmers      &        4013         &        4004         &        4004         &        4004         \\
Farmer-Crops        &        9554         &        9536         &        9536         &        9536         \\

	\addlinespace
	\addlinespace
	\multicolumn{5}{c}{\emph{Panel B. Total Value of Output, imputed ('000 INR per Ha)}} \\
	\addlinespace
	Well depth (1 sd = 187 feet)&       -0.51         &       -2.95\sym{**} &       -8.64\sym{***}&       -6.38\sym{**} \\
                    &      (1.05)         &      (1.33)         &      (2.62)         &      (2.87)         \\
Toposequence        &                     &         Yes         &         Yes         &         Yes         \\
Soil quality controls&                     &         Yes         &         Yes         &         Yes         \\
Subdivisional effects&                     &         Yes         &                     &                     \\
Plot size effects   &                     &         Yes         &                     &                     \\
                    &                     &                     &                     &                     \\
Mean dep. var       &        65.4         &        65.4         &        65.4         &        65.4         \\
Candidate Instruments&                     &                     &                     &                     \\
Instruments Selected&                     &                     &                     &                     \\
Unique Farmers      &                     &                     &                     &                     \\
Farmer-Crops        &        9290         &        9272         &        9272         &        9272         \\

	\bottomrule
	\multicolumn{5}{p{0.96\hsize}}{\footnotesize The table reports coefficients from regressions of agricultural output measures on well depth and controls. The data is from the main agricultural household survey and the observations are at the farmer-by-crop level. The dependent variable changes in each panel. In Panel A, the dependent variable is yield (quintals per Ha). In Panel B, the dependent variable is the value of output (INR per Ha), where the price for each crop is taken to be the median of the price reported at the SDO level. Well depth is the reported depth of a given farmer's well. Toposequence includes controls for elevation and slope. Subdivisional effects are dummy variables for each of the six sub-divisional offices of the distribution company from which farmers were sampled. Plot size effects are dummy variables indicating the plot size decile for each farmer-crop based on its plot area. Standard errors are clustered at the feeder level, the block within which farmers were randomly selected. The statistical significance of a coefficient at certain thresholds is indicated by  * $p < 0.10$, ** $p < 0.05$, *** $p < 0.01$.} \\
\end{tabular*}
\end{table}
