\documentclass{article}

\usepackage[utf8]{inputenc}
\usepackage{booktabs}
\usepackage{multirow}
\usepackage{makecell}
\usepackage{tabularx}
\usepackage{booktabs}
\usepackage{pdflscape}
\usepackage{caption} 
\usepackage{amsmath}
\usepackage{graphicx}
\usepackage{amssymb}
\usepackage{adjustbox}
\usepackage{tablefootnote}
\usepackage[capposition=top]{floatrow}
%\usepackage{subcaption}
\usepackage{dcolumn}
\usepackage{subfigure}
\usepackage{soul}
\usepackage{array}
\usepackage{lscape}
\usepackage{tabu}
\usepackage[hidelinks=true]{hyperref}
\hypersetup{
	colorlinks   = true, %Colours links instead of ugly boxes
	urlcolor     = blue, %Colour for external hyperlinks
	linkcolor    = blue, %Colour of internal links
	citecolor   = red %Colour of citations
}
\captionsetup[table]{skip=10pt}
\usepackage[margin=0.8in]{geometry}
\numberwithin{table}{section}
\renewcommand{\thefigure}{\arabic{section}.\arabic{figure}}

\DeclareMathOperator{\Cov}{Cov}
\DeclareMathOperator{\Var}{Var}

% Allow more floats to fit on a page
\renewcommand\floatpagefraction{.9}
\renewcommand\topfraction{.9}
\renewcommand\bottomfraction{.9}
\renewcommand\textfraction{.1}   

% New column types for fixed-width
\newcolumntype{L}[1]{>{\raggedright\let\newline\\\arraybackslash\hspace{0pt}}m{#1}}
\newcolumntype{C}[1]{>{\centering\let\newline\\\arraybackslash\hspace{0pt}}m{#1}}
\newcolumntype{R}[1]{>{\raggedleft\let\newline\\\arraybackslash\hspace{0pt}}m{#1}}

% Set table/figure paths
\makeatletter
\def\input@path{{tables/}{tables/static/}}
\makeatother

\graphicspath{{figures/}{figures/static/}}

\title{Rationing the Commons \\ Exhibits}

\begin{document}
	
	\maketitle
	
	\tableofcontents
	
	\clearpage
	\addcontentsline{toc}{section}{\listfigurename}
	\listoffigures
	\addcontentsline{toc}{section}{\listtablename}
	\listoftables
	
\clearpage	
\section{Paper exhibits}

\begin{figure}[h]
	\centering \caption{Rationing of Power Supply in Rajasthan\label{fig:rationing}}
	\subfiguretopcaptrue
	\subfigure[Hours of Supply]{
		\includegraphics[width=0.47\linewidth]{daily_electricity_supply.pdf} 
	}
	\subfigure[Hours of Use]{
		\includegraphics[width=0.47\linewidth]{daily_electricity_usage.pdf} 
	}
	\begin{tabular*}{1.0\textwidth}{c}
		\multicolumn{1}{p{1.0\hsize}}{\footnotesize The figure shows power rationing using data from our agricultural survey in Rajasthan. Panel A shows the distribution of the average hours of supply per day during the Rabi season of 2016-2017. Panel B shows the distribution of the average hours of use over the season.}\\
	\end{tabular*}
\end{figure}

%\begin{figure}[htbp]
%	\centering \caption{Variation in Well Depth\label{fig:firstStage}}
%	\subfigure[Well Depth in Dug]{
%		\includegraphics[width = 0.8\linewidth]{together_D.pdf}
%	}
%	\subfigure[Well Depth in Hindoli and Nainwa]{
%		\includegraphics[width = 0.8\linewidth]{together_HN.pdf}
%	}
%	\subfigure[Well Depth in Kotputli, Bansur, and Mundawar]{
%		\includegraphics[width = 0.8\linewidth]{together_KBM.pdf}	
%	}
%	\begin{tabular*}{1.0\textwidth}{c}
%		\multicolumn{1}{p{1.0\hsize}}{\footnotesize The figure shows the variation in well depth for the three groups of subdivisions in our sample. Each panel shows a different area: Dug subdivision (panel A), Hindoli and Nainwa subdivisions (panel B) and Kotputli, Bansur and Mundawar subdivisions (panel C). Within each panel, the map on the left shows the actual depth of wells as reported by farmers, against the scale at right. The map on the right shows the depth of wells that is predicted based on geological factors. The set of geological factors used as instruments is described in Section~\ref{sec:data} in brief and Appendix~\ref{sec:appendixData} gives the rationale for these factors.}\\
%	\end{tabular*}
%\end{figure}


\begin{figure}[htbp]
	\centering \caption{Optimality of ration\label{fig:optimalityOfRation}}
	\includegraphics[width = 0.9\linewidth]{fig_optimal_ration}
	\begin{tabular*}{1.0\textwidth}{c}
		\multicolumn{1}{p{1.0\hsize}}{\footnotesize The figure compares the marginal benefit and marginal cost of a one hour increase in the ration on average across all farmers. The estimates come from the marginal analysis presented in Section~\ref{sec:marginalAnalysis}. The marginal benefit is derived from our estimate in Table~\ref{tab:ivProfitsDepth}, column 3 using the calculation shown in Table~\ref{tab:optimalRation}, column 1. The marginal cost is similarly calculated in Table~\ref{tab:optimalRation}, column 2. The left-hand axis gives the marginal benefit or cost in units of INR thousand per Ha per hour increase in the ration. The right-hand axis gives the marginal benefit or cost as a percentage of the annual household income of agricultural households. The whiskers show 90\% confidence intervals for each estimate.}
	\end{tabular*}
\end{figure}

\begin{figure}[htbp]
\centering\caption{Distribution of productivity across farmer-crops\label{fig:tfpDistribution}}
\includegraphics[width = 0.65\linewidth]{fig_tfp_distribution.pdf}
\begin{tabular*}{1.0\hsize}{c}
	\multicolumn{1}{p{1.0\hsize}}{\footnotesize The figure shows the distribution of productivity from the estimated production function. The dotted line gives the distribution of the raw, total factor productivity residual from estimation of equation \ref{eq:productionEstimating}. The solid line gives the distribution of total factor productivity after correcting for measurement error in output and inputs.}\\
	%    \multicolumn{1}{p{1.0\hsize}}{\footnotesize The figure shows the distribution of productivity from the estimated production function. The dotted line gives the distribution of the raw, total factor productivity residual from estimation of equation \ref{eq:productionEstimating}, called $\widehat{TFP}_a$ in the text. The dashed line gives the distribution of total factor productivity after correcting for measurement error in output and inputs, in a model with both total factor productivity shocks and factor-specific shocks, as in \citet{Gollin2017} ($\widehat{TFP}_b$). The solid line gives the distribution of total factor productivity after correcting for measurement error in output and inputs, as defined in the text ($\widehat{TFP}_c$). The distribution of $\widehat{TFP}_b$ is wider that the distribution of $\widehat{TFP}_c$ because the \citet{Gollin2017} model allows for factor-specific productivity shocks to contribute to unobserved productivity and hence dispersion in output, whereas our model does not, and attributes this variation to factor measurement error.}\\
\end{tabular*}
\end{figure}

\begin{figure}[htbp]
	\centering\caption{Shadow cost of the status quo ration\label{fig:shadow}}
	\includegraphics[width = 0.80\linewidth]{fig_shadow_value_ration}
	\begin{tabular*}{1.0\hsize}{c}
		\multicolumn{1}{p{1.0\hsize}}{\footnotesize The figure shows the distribution of the shadow cost of the status quo ration, of 6 hours of power per day, across farmer-plots, plus the nominal electricity price of INR 0.9 per kWh. The sum of the shadow cost and the nominal monetary power cost is therefore the overall cost of power faced by each farmer. The shadow cost of the ration gives the marginal benefit of power use and is calculated as the additional price of power that would induce each farmer to use the rationed amount of power on their plot if they were unconstrained. The two dashed vertical lines show benchmarks on the social cost of power use. The line on the left shows the private marginal cost of energy supply $c_E = 6.2$ INR/kWh. The line of the right shows the average social marginal cost of power use, i.e.~the Pigouvian price of power.}\\
	\end{tabular*}
\end{figure}

\begin{figure}[htbp]
	\centering\caption{Change in Profit Due to Pigouvian Reform by Plot Size\label{fig:deltaProfitLand}}
	\includegraphics[width = 0.8\linewidth]{fig_pigouvian_redistribution_profit_byLand_panel3.pdf}
	\begin{tabular*}{1.0\hsize}{c}
		\multicolumn{1}{p{1.0\hsize}}{\footnotesize The figure shows the mean, across farmers and crops, of the change in profit from a reform that replaces the status quo rationing regime with a Pigouvian regime that prices power at social cost, \emph{without} any compensating transfers to farmers. A negative value therefore means profits decline under the Pigouvian regime and a positive value that they increase. The three separate curves show the mean change in profit, plotted by plot size, for farmer-crops in the bottom quartile of the productivity distribution (dashed line), the top quartile (dashed and dotted line), and all farmers (solid line). The data is smoothed using a local linear regression.}\\
	\end{tabular*}
\end{figure}


% Table 1 : Farmer-crop summary statistics
%   \ref{tab:summaryStatistics}
\renewcommand{\tabcolsep}{3pt}
\begin{table}[htbp]\centering
\def\sym#1{\ifmmode^{#1}\else\(^{#1}\)\fi}
\caption{Summary statistics on farmer survey sample\label{tab:sum_stats}}
\begin{tabular}{l*{1}{cccccc}}
\toprule
                    &        Mean&    Std. dev&        25th&      Median&        75th&Obs.\\
                    & 		(1)     &	(2)         &(3) &(4) &(5)&(6) \\
\midrule
\addlinespace
\multicolumn{7}{c}{\emph{Panel A: Farmer-level crops and water access}} \\
\addlinespace
Crops grown (number)&        2.30&        1.15&           2&           2&           3&        4259\\
Pump capacity (HP, total)&        12.5&        8.28&        7.50&          10&          15&        4020\\
Well depth (feet)   &       287.8&       186.5&         150&         275&         390&        4020\\

\addlinespace
\multicolumn{7}{c}{\emph{Panel B: Farmer-crop output and profit}} \\
\addlinespace
Yield (quintals/ha) &        46.3&       109.5&        12.3&        24.7&        49.3&        9560\\
Output (quintals)   &        22.2&        42.2&           5&          12&          25&        9564\\
Total value of output (INR '000s/ha)&        65.4&       101.8&        37.0&        59.2&        78.9&        9296\\
Cash profit (INR '000s/ha)&       -13.5&        47.1&       -33.6&       -14.1&        15.4&        3254\\
Total profit (INR '000s/ha)&       -5.07&       118.7&       -25.1&        0.25&        23.6&        8997\\
\hspace{1 em} Own labor at MNREGA wage&        1.49&       117.7&       -19.0&        6.87&        29.0&        8997\\

\addlinespace
\multicolumn{7}{c}{\emph{Panel C: Farmer-crop input quantities}} \\
\addlinespace
Land (ha)           &        0.65&        0.72&        0.24&        0.41&        0.81&        9564\\
Water ('000 ltr)    &      1469.4&      2468.9&       401.4&       828.1&      1685.8&        9544\\
Labor (worker-days) &        57.9&        53.2&          22&          40&          75&        9564\\

\addlinespace
\multicolumn{7}{c}{\emph{Panel D: Farmer-crop input expenditures}} \\
\addlinespace
Capital ('000 INR)  &        17.0&        16.2&        6.71&        12.0&        21.3&        9255\\
Labour ('000 INR)   &        17.3&        17.5&        5.95&        11.5&        22.4&        9564\\
Electricity (subsidized) ('000 INR)&        1.19&        0.39&        0.85&        1.27&        1.27&        9564\\

\addlinespace
%\multicolumn{7}{c}{\emph{Panel D: Groundwater conditions and adaptation}} \\
%\addlinespace
%Parcel leveled      &        0.15&        0.36&           0&           0&           0&       10700\\
Parcel sprinker irrigated&        0.33&        0.47&           0&           0&           1&       14558\\
Furrow/flood irrigation used&        0.28&        0.45&           0&           0&           1&       14677\\
Parcel under-irrigated&        0.24&        0.43&           0&           0&           0&       10700\\

%High-yielding variety&        0.62&        0.49&           0&           1&           1&        8711\\

\bottomrule
\multicolumn{7}{p{0.98\hsize}}{\footnotesize The table provides summary statistics on variables from the Rajasthan farmer survey. All observations in Panel A are at the farmer level. Observations in other panels are at the farmer-crop level. Panel A describes farmer level attributes: the number of crops grown, the total pump capacity and average well depth. Panel B shows variables on output and profit: the total physical quantity of output, the yield (output/area), the total value of output, and four measures of profit. The four measures are cash profit, which was directly reported by farmers, total profit, which is cash profit plus the value of own consumption less the imputed value of own expenses, profit where household labor is valued at the National Rural Employment Guarantee Act rate, and profit where household labor is not counted as a cost. Panel B gives physical input quantities of land, water and labor. Capital is heterogeneous so there is no relevant physical measure of capital. Panel C gives monetary input expenditures.}
\end{tabular}
\end{table}


% Table 2 : IV regressions for profit 
%   \ref{tab:ivProfitsDepth}
\renewcommand{\tabcolsep}{8pt}
\begin{table}[htbp]\centering
\def\sym#1{\ifmmode^{#1}\else\(^{#1}\)\fi}
\caption{Hedonic regressions of profit on well depth\label{tab:ivProfitsDepth}}

\begin{tabular*}{1\hsize}{@{\hskip\tabcolsep\extracolsep\fill}l*{4}{D{.}{.}{-1}}}
	\toprule
	&\multicolumn{1}{c}{OLS}&\multicolumn{1}{c}{OLS}&\multicolumn{1}{c}{IV-PDS}&\multicolumn{1}{c}{IV-PDS}\\
	&\multicolumn{1}{c}{(1)}&\multicolumn{1}{c}{(2)}&\multicolumn{1}{c}{(3)}&\multicolumn{1}{c}{(4)}\\
	\midrule
	\addlinespace
	\multicolumn{5}{c}{\emph{Panel A. Total Profit ('000 INR per Ha)}} \\
	\addlinespace
	Well depth (1 sd = 187 feet)&        0.68         &       -2.74\sym{*}  &       -8.44\sym{***}&       -6.95\sym{***}\\
                    &      (1.25)         &      (1.56)         &      (2.41)         &      (2.63)         \\
Toposequence        &                     &         Yes         &         Yes         &         Yes         \\
Soil quality controls&                     &         Yes         &         Yes         &         Yes         \\
Subdivisional effects&                     &         Yes         &         Yes         &         Yes         \\
Plot size effects   &                     &         Yes         &         Yes         &         Yes         \\
                    &                     &                     &                     &                     \\
Mean dep. var       &       -5.06         &       -5.06         &       -5.06         &       -5.06         \\
Candidate Instruments&                     &                     &         419         &        1728         \\
Instruments Selected&                     &                     &          16         &          18         \\
Unique Farmers      &        4008         &        3999         &        3999         &        3999         \\
Farmer-Crops        &        8991         &        8973         &        8973         &        8973         \\

	\addlinespace
	\multicolumn{5}{c}{\emph{Panel B. Cash Profit, reported ('000 INR per Ha)}} \\
	\addlinespace
	Well depth (1 sd = 187 feet)&       -3.46\sym{***}&       -2.94\sym{***}&       -10.7\sym{**} &       -18.1\sym{***}\\
                    &      (0.82)         &      (1.03)         &      (5.03)         &      (6.06)         \\
Toposequence        &                     &         Yes         &         Yes         &         Yes         \\
Soil quality controls&                     &         Yes         &         Yes         &         Yes         \\
Subdivisional effects&                     &         Yes         &         Yes         &         Yes         \\
Plot size effects   &                     &         Yes         &         Yes         &         Yes         \\
                    &                     &                     &                     &                     \\
Mean dep. var       &       -13.5         &       -13.5         &       -13.5         &       -13.5         \\
Candidate Instruments&                     &                     &         419         &        1728         \\
Instruments Selected&                     &                     &           9         &           6         \\
Unique Farmers      &        2127         &        2121         &        2121         &        2121         \\
Farmer-Crops        &        3253         &        3243         &        3243         &        3243         \\

	\bottomrule
	\multicolumn{5}{p{0.96\hsize}}{\footnotesize The table reports coefficients from regressions of agricultural profit measures on well depth and controls. The data is from the main agricultural household survey and the observations are at the farmer-by-crop level. The dependent variable changes in each panel. In Panel A, the dependent variable is total profit, which includes the value of the farmer's own consumption (INR per Ha). Where no profit is reported in cash, and the farmer keeps all the output for own consumption, we impute the total profit variable by adding the value of output consumed \and subtracting the input costs associated with the output. The price used in this imputation for each crop is taken to be the median market price for the crop reported at the SDO level.  In Panel B, the dependent variable is reported cash profit (INR per Ha). Well depth is the reported depth of a given farmer's well. Toposequence includes controls for elevation and slope. Subdivisional effects are dummy variables for each of the six sub-divisional offices of the distribution company from which farmers were sampled. Plot size effects are dummy variables indicating the plot size decile for each farmer-crop based on its plot area. Standard errors are clustered at the feeder level, the block within which farmers were randomly selected. The statistical significance of a coefficient at certain thresholds is indicated by  * $p < 0.10$, ** $p < 0.05$, *** $p < 0.01$.} \\
\end{tabular*}

\end{table}


% Table 3: Production function coefficients
%   \ref{tab:prodFunc}
\renewcommand{\tabcolsep}{8pt}
% \input{exhibits/standard_production_function_paper}
\begin{table}[!ht]
	\centering
		\caption{Production Function Estimates} \label{tab:prodFunc}
	\def\sym#1{\ifmmode^{#1}\else\(^{#1}\)\fi}
	\begin{tabular}{lrrrr}
		\toprule
\multicolumn{1}{l}{\emph{Dependent variable}} &\multicolumn{4}{c}{log(Value of output)} \\
\cmidrule(lr){2-5}
&\multicolumn{1}{c}{OLS}&\multicolumn{1}{c}{2SLS}&\multicolumn{1}{c}{2SLS}&\multicolumn{1}{c}{2SLS}\\
\cmidrule(lr){2-2}\cmidrule(lr){3-3}\cmidrule(lr){4-4}\cmidrule(lr){5-5}
\multicolumn{1}{l}{\emph{Endogenous inputs:}}&\multicolumn{1}{c}{ }&\multicolumn{1}{c}{Water}&\multicolumn{1}{c}{All}&\multicolumn{1}{c}{All}\\
&\multicolumn{1}{c}{(1)}&\multicolumn{1}{c}{(2)}&\multicolumn{1}{c}{(3)}&\multicolumn{1}{c}{(4)}\\
		\midrule
		     log(Water)&0.04\sym{***}&0.18\sym{***}&0.16\sym{***}&0.18\sym{***}\\
		               &(0.010)&(0.063)&(0.061)&(0.044)\\
		 \addlinespace 
		      log(Land)&0.54\sym{***}&0.49\sym{***}&0.50\sym{***}&0.50\sym{***}\\
		               &(0.040)&(0.047)&(0.057)&(0.059)\\
		 \addlinespace 
		     log(Labor)&0.16\sym{***}&0.12\sym{***}&0.22&0.22\\
		               &(0.025)&(0.030)&(0.157)&(0.180)\\
		 \addlinespace 
		   log(Capital)&0.34\sym{***}&0.34\sym{***}&0.31\sym{**}&0.31\sym{*}\\
		               &(0.032)&(0.033)&(0.150)&(0.169)\\
		 \addlinespace 
		 \addlinespace 
		   Toposequence&\emph{Yes}&\emph{Yes}&\emph{Yes}&\emph{Yes}\\
		   Soil quality&\emph{Yes}&\emph{Yes}&\emph{Yes}&\emph{Yes}\\
		Subdivisional effects&\emph{Yes}&\emph{Yes}&\emph{Yes}&\emph{Yes}\\
		 \addlinespace 
		 \addlinespace 
		  Mean dep. var&3.24&3.24&3.24&3.24\\
		        Farmers&3998&3998&3998&3998\\
		   Farmer-crops&8711&8711&8711&8711\\
		\bottomrule
	\multicolumn{5}{p{0.55\hsize}}{\footnotesize The table reports estimates of the production function. The dependent variable is the log of the total value of agricultural output. The independent variables are the logs of productive inputs, water, land, labor and capital, as well as exogenous control variables. All specifications include as controls subdivision fixed effects, as described in the notes of Table \ref{tab:ivProfitsDepth}, toposequence variables for elevation and slope, and soil quality measured at the village level (acidity/alkalinity of the soil along with variables measuring the level of eight minerals). The columns vary in the method of estimation and what variables are treated as endogenous. Column 1 shows OLS estimates. Column 2 shows instrumental variables estimates treating only water as endogenous and using as instruments only geological factors. Column 3 shows instrumental variables estimates treating all four inputs as endogenous. The first stage results for the column 3 specification are reported in Appendix~\ref{sec:appendixRobustness}, Table~\ref{tab:firstStageProdFunc}. Column 4 takes the column 3 estimates and calibrates the elasticity of output with respect to water to match the marginal benefit of relaxing the ration by one hour, as reported in Table~\ref{tab:optimalRation}, column 1, panel A. Columns 1 to 3 report analytic standard errors clustered at the level of the feeder, the primary sampling unit. Column 4 reports cluster-bootstrapped standard errors, also clustered at the feeder level, to account for uncertainty in the estimated marginal benefit of relaxing the ration. Statistical significance is indicated by \sym{*} $ p < 0.10$, \sym{**} $ p < 0.05$, \sym{***} $ p < 0.01$.}\\
	\end{tabular}
\end{table}


% Table 4: mean counterfactual outcomes
%   \ref{tab:cfOutcomes}
\renewcommand{\tabcolsep}{8pt}
\begin{table}[!ht]
	\centering
		\caption{Counterfactual Production and Social Surplus\label{tab:cfOutcomes}}
\begin{tabular}{lrrrr}
		\toprule
& \multicolumn{2}{c}{Rationing} & \multicolumn{2}{c}{Pricing} \\
\cmidrule(lr){2-3} \cmidrule(lr){4-5}               &Status quo&Optimal&Private cost&Pigouvian\\
&\multicolumn{1}{c}{(1)}&\multicolumn{1}{c}{(2)}&\multicolumn{1}{c}{(3)}&\multicolumn{1}{c}{(4)}\\
		\midrule
		 \addlinespace 
\multicolumn{5}{c}{\emph{A. Profits and social surplus}}\\
		 \addlinespace 
		Profit (INR 000s)&18.72&16.64&19.39&12.90\\
		~~~~$-$ Unpriced power cost (INR 000s)&5.35&4.21&0.00&-4.64\\
		~~~~$-$ Water cost (INR 000s)&5.33&4.18&9.24&4.94\\
\cmidrule(lr){1-1}		Surplus (INR 000s)&8.04&8.25&10.15&12.60\\
		 \addlinespace 
\multicolumn{5}{c}{\emph{B. Input use}}\\
		 \addlinespace 
		      Land (Ha)&0.69&0.69&0.69&0.69\\
		Labor (person-days)&54.81&54.81&54.81&54.81\\
		Capital (INR 000s)&16.58&15.54&20.79&17.81\\
		Water (liter 000s)&1590.65&1248.65&2758.34&1475.47\\
		~~~~Power (kWh per season)&1010.20&793.95&1517.26&768.16\\
		~~~~Hours of use (per day)&5.95&4.67&10.63&5.84\\
		 \addlinespace 
\multicolumn{5}{c}{\emph{C. Output and productivity}}\\
		 \addlinespace 
		Output (INR 000s)&52.78&49.46&66.15&56.68\\
		~~~~Gain in output from status quo (pp)&&  -6&  25&   7\\
		~~~~Gain in output due to input use (pp)&&  -6&  19&   1\\
		~~~~Gain in output due to productivity (pp)&&  -0&   7&   7\\
		$\Cov(\Omega_{Eit},W_{it}^{\alpha_W})$&-0.04&-0.04&0.24&0.25\\
		\bottomrule
\multicolumn{5}{p{0.85\hsize}}{\footnotesize The table shows the outcomes of counterfactual policy regimes with respect to farmer profit, external costs and social surplus. The columns show different policy regimes: the status quo rationing regime, with a ration of 6 hours and a price of INR 0.90 per kWh, a private cost regime, where power is priced at its private marginal cost of INR 6.2 per kWh, and a Pigouvian regime where power is priced at marginal social cost. The rows show the outcome variables in each regime. All outcome variables, except where noted, are mean values at the farmer-by-crop level, where the average farmer plants 2.3 crops. Panel C shows output and the change in output, in percentage points, relative to the status quo value under rationing. Row 3 gives the change in output that would have been achieved from a proportional change in input use for all farmers, equal to the aggregate proportional change in input use in each scenario relative to column 1. Row 4 then gives the residual change in output due to increases in aggregate productivity from the input reallocation. Finally, row 5 reports the covariance between $\Omega_{Eit}$ and the contribution of water input to production.} \\
	\end{tabular}
\end{table}


% Table 5: distributional effects of Pigouvian reform
%   \ref{tab:cfDistribution}
\renewcommand{\tabcolsep}{9pt}
\begin{table}[!ht]
	\centering
		\caption{Distributional Effects of Pigouvian Reform\label{tab:cfDistribution}}
\begin{tabular}{lrrrrr}
		\toprule
& Rationing & \multicolumn{4}{c}{Pigouvian} \\
\cmidrule(lr){2-2} \cmidrule(lr){3-6}\multicolumn{1}{c}{Transfers:}&\multicolumn{1}{c}{None}&\multicolumn{1}{c}{None}&\multicolumn{1}{c}{Flat}&\multicolumn{1}{c}{Pump}&\multicolumn{1}{c}{Land}\\
&\multicolumn{1}{c}{(1)}&\multicolumn{1}{c}{(2)}&\multicolumn{1}{c}{(3)}&\multicolumn{1}{c}{(4)}&\multicolumn{1}{c}{(5)}\\
		\midrule
		 \addlinespace 
\multicolumn{6}{c}{\emph{A. Inequality under different transfer schemes}}\\
		 \addlinespace 
		Mean profit (INR 000s)&42.64&30.42&30.42&30.42&30.42\\
		~~~~$+$ Mean transfer (INR 000s)&0.00&0.00&22.23&22.23&22.23\\
\cmidrule(lr){1-1}		Mean net profit (INR 000s)&42.64&30.42&52.65&52.65&52.65\\
		 \addlinespace 
		Std dev net profit (INR 000s)&73.98&83.08&84.86&85.91&88.62\\
		 \addlinespace 
\multicolumn{6}{c}{\emph{B. Change from rationing regime due to reform}}\\
		 \addlinespace 
		 Share who gain&  &0.10&0.75&0.67&0.61\\
		 \multicolumn{6}{l}{\emph{Conditional on gain in profit:}}\\
		~~~~Mean ex ante profit&  &135.68&35.90&39.32&42.54\\
		~~~~Mean change in net profit&  &27.47&17.52&20.08&23.66\\
		~~~~Mean land (Ha)&  &3.43&1.46&1.51&1.61\\
		~~~~Mean depth (feet)&  &213.16&277.72&292.69&275.57\\
		~~~~Mean productivity (percentile)&  &55.65&46.69&49.52&45.98\\
		 \addlinespace 
		 Share who lose&  &0.90&0.25&0.33&0.39\\
		 \multicolumn{6}{l}{\emph{Conditional on loss in profit:}}\\
		~~~~Mean ex ante profit&  &32.39&62.65&49.44&42.79\\
		~~~~Mean change in net profit&  &-16.59&-12.27&-10.64&-11.65\\
		~~~~Mean land (Ha)&  &1.30&1.67&1.51&1.34\\
		~~~~Mean depth (feet)&  &296.12&318.06&278.02&307.43\\
		~~~~Mean productivity (percentile)&  &49.93&61.81&52.51&57.68\\
		\bottomrule
\multicolumn{6}{p{0.80\hsize}}{\footnotesize The table shows the distributional impacts of Pigouvian reform on farmer profits. The columns show results for different policy regimes: column 1 is the status quo rationing regime and columns 2 through 4 show regimes with Pigouvian pricing. The Pigouvian regimes differ in the transfers made to farmers and how those transfers are conditioned. In column 2 onwards, the transfer policies are: no transfers, flat (uniform) transfers, transfers pro rata based on pump capacity, and transfers pro rata based on land size. The rows in Panel A show summary statistics on the level of profits under different regimes. The rows in Panel B show summary statistics on the changes in profits between the status quo rationing regime (column 1) and the respective Pigouvian regimes (columns 2 through 5)} \\
	\end{tabular}
\end{table}


\clearpage
\section{Appendix Exhibits}	

\begin{figure}[htbp]
	\centering
	\caption{Extensive margin\label{fig:sanctionedLoadBinds}}
	\subfiguretopcaptrue
	\subfigure[Distribution of waiting times to acquire pump connection]{
		\includegraphics[width=0.47\linewidth]{hist_waiting_times.pdf} 
	}
	\subfigure[Ratio of actual pump load to sanctioned load]{
		\includegraphics[width=0.47\linewidth]{hist_load_ratio.pdf} 
	}
	\begin{tabular*}{1.0\textwidth}{c}
		\multicolumn{1}{p{1.0\hsize}}{\footnotesize This figure provides empirical evidence that the ration binds on all dimensions. Panel A shows the distribution of the ratio of the actual pump load in our farmer survey to the sanctioned load, which is the load the farmer is allowed by the government to have by the terms of their electricity connection. The modal farmer reports that they use exactly the sanctioned load and relatively few farmers have actual pump loads above the sanctioned load. Panel B shows the distribution of wait times in years for acquiring an agricultural pump connection from the power utility company in Hindoli and Mundawar, two of the subdivisional areas in our sample. The data consists of application and approval dates of connection requests from farmers who applied for a pump between 2010 and 2014.}\\
	\end{tabular*}
\end{figure}

\clearpage

\renewcommand{\tabcolsep}{12pt}
\begin{table}[htbp]\centering
	\caption{Definition of candidate instrument sets\label{tab:instruments}}
		\begin{tabular}{p{4cm}p{1.5cm}p{1.5cm}p{1.5cm} p{1.5cm} p{1.5cm}}
			\toprule
			& Fractures         & Rocks         & Aquifers     & Main         & Large        \\ \midrule
			Fractures                                       & \emph{Yes} &              & \emph{Yes} & \emph{Yes} & \emph{Yes} \\ 
			Rock type                                       &              & \emph{Yes} & \emph{Yes} & \emph{Yes} & \emph{Yes} \\ 
			Rock share                                      &              & \emph{Yes} & \emph{Yes} & \emph{Yes} & \emph{Yes} \\
			Aquifer type                                    &              &              & \emph{Yes} & \emph{Yes} & \emph{Yes} \\
			$\text{Fractures}^2$                            &              &              &              & \emph{Yes} & \emph{Yes} \\
			$\text{Rock share}^2$                           &              &              &              & \emph{Yes} & \emph{Yes} \\
			$\text{Fractures} \times \text{Rock type}$      &              &              &              & \emph{Yes} & \emph{Yes} \\
			$\text{Fractures} \times \text{Rock share}$     &              &              &              & \emph{Yes} & \emph{Yes} \\
			$\text{Fractures}^2 \times \text{Rock type}$    &              &              &              &              & \emph{Yes} \\
			$\text{Fractures}^2 \times \text{Rock share}$   &              &              &              &              & \emph{Yes} \\
			$\text{Rock share}^2 \times \text{Fractures}$   &              &              &              &              & \emph{Yes} \\
			$\text{Rock share}^2 \times \text{Fractures}^2$ &              &              &              &              & \emph{Yes} \\ \midrule
			Size of instrument set &3 &130 &153 &419 &1728 \\ \bottomrule
			\multicolumn{6}{p{0.85\hsize}}{\footnotesize This table defines the instruments contained in our candidate geological instrument sets. Broadly, the geological variables consist of data on rock fractures, rock types, rock shares and aquifer types. Data on fractures encapsulates the distance between a farmer's location and the nearest water conductive fracture and the total length of such fractures in a radius of one and five kilometres around the farmer's location. Data on rock types consists of sixty-five dummy variables which indicate the type of rock  at the farmer's precise location. Rock share variables capture the share of a given rock type in a five kilometre radius around the farmer's precise position. Aquifer types are dummy variables which indicate the presence of any of twenty types of aquifers that are located at the farmer's precise location. The instrument sets are composed of these basic variables and their interactions. }
		\end{tabular}
	
		
\end{table}

\renewcommand{\tabcolsep}{8pt}
\begin{table}[htbp]\centering
\def\sym#1{\ifmmode^{#1}\else\(^{#1}\)\fi}
\caption{First Stage: Well depth on instruments\label{tab:robustnessFirstStage}}
%\adjustbox{max width=\textwidth}{%
\begin{tabular*}{\hsize}{@{\hskip\tabcolsep\extracolsep\fill}l*{5}{D{.}{.}{-1}}}
\toprule
&\multicolumn{1}{c}{IV-2SLS}&\multicolumn{1}{c}{IV-PDS}&\multicolumn{1}{c}{IV-PDS}&\multicolumn{1}{c}{IV-PDS}&\multicolumn{1}{c}{IV-PDS}\\
\cmidrule(lr){2-2}\cmidrule(lr){3-3}\cmidrule(lr){4-4}\cmidrule(lr){5-5}\cmidrule(lr){6-6}
&\multicolumn{1}{c}{Fractures}&\multicolumn{1}{c}{Rock}&\multicolumn{1}{c}{Aquifers}&\multicolumn{1}{c}{Main}&\multicolumn{1}{c}{Large}\\
&\multicolumn{1}{c}{(1)}&\multicolumn{1}{c}{(2)}&\multicolumn{1}{c}{(3)}&\multicolumn{1}{c}{(4)}&\multicolumn{1}{c}{(5)}\\
\midrule \\
Fractures           &         Yes&            &            &            &            \\
Rock shares         &            &         Yes&         Yes&         Yes&         Yes\\
Rock types          &            &         Yes&         Yes&         Yes&         Yes\\
Aquifer types       &            &            &         Yes&         Yes&         Yes\\
$\text{Fractures}^2$&            &            &            &            &            \\
$\text{Rock shares}^2$&            &            &            &         Yes&            \\
$\text{Fractures} \times \text{Rock shares}$&            &            &            &         Yes&         Yes\\
$\text{Fractures}^2 \times \text{Rock shares}$&            &            &            &            &         Yes\\
$\text{Fractures} \times \text{Rock shares}^2$&            &            &            &            &            \\
$\text{Fractures}^2 \times \text{Rock shares}^2$&            &            &            &            &         Yes\\
                    &            &            &            &            &            \\
RMSE                &       148.4&       417.8&       427.2&       423.8&       364.5\\
F                   &       137.3&       104.5&        98.3&        86.8&        84.6\\
Candidate Instruments&            &         130&         153&         419&        1728\\
Instruments Selected&            &            &            &            &            \\
Unique Farmers      &        3999&        3999&        3999&        3999&        3999\\
Farmer-Crops        &        9540&        8973&        8973&        8973&        8973\\

\bottomrule
\multicolumn{6}{p{1\hsize}}{\footnotesize * $p < 0.10$, ** $p < 0.05$, *** $p < 0.01$.} \\
\end{tabular*}
\end{table}

\renewcommand{\tabcolsep}{10pt}
\begin{table}[htbp]\centering
\def\sym#1{\ifmmode^{#1}\else\(^{#1}\)\fi}
\caption{Hedonic regressions of yield on well depth\label{tab:ivYieldDepth}}

\begin{tabular*}{1\hsize}{@{\hskip\tabcolsep\extracolsep\fill}l*{4}{D{.}{.}{-1}}}
	\toprule
	&\multicolumn{1}{c}{OLS}&\multicolumn{1}{c}{OLS}&\multicolumn{1}{c}{IV-PDS}&\multicolumn{1}{c}{IV-PDS}\\
	&\multicolumn{1}{c}{(1)}&\multicolumn{1}{c}{(2)}&\multicolumn{1}{c}{(3)}&\multicolumn{1}{c}{(4)}\\
	\midrule
	
	\addlinespace
	\multicolumn{5}{c}{\emph{Panel A. Yield (quintals per Ha)}} \\
	\addlinespace
	Well depth (1 sd = 187 feet)&       -7.99\sym{***}&       -2.15         &       -6.86\sym{***}&       -3.83         \\
                    &      (1.12)         &      (1.40)         &      (2.66)         &      (2.83)         \\
Toposequence        &                     &         Yes         &         Yes         &         Yes         \\
Soil quality controls&                     &         Yes         &         Yes         &         Yes         \\
Subdivisional effects&                     &         Yes         &         Yes         &         Yes         \\
Plot size effects   &                     &         Yes         &         Yes         &         Yes         \\
                    &                     &                     &                     &                     \\
Mean dep. var       &        46.3         &        46.3         &        46.3         &        46.3         \\
Candidate Instruments&                     &                     &         419         &        1728         \\
Instruments Selected&                     &                     &          15         &          16         \\
Unique Farmers      &        4013         &        4004         &        4004         &        4004         \\
Farmer-Crops        &        9554         &        9536         &        9536         &        9536         \\

	\addlinespace
	\addlinespace
	\multicolumn{5}{c}{\emph{Panel B. Total Value of Output, imputed ('000 INR per Ha)}} \\
	\addlinespace
	Well depth (1 sd = 187 feet)&       -0.51         &       -2.95\sym{**} &       -8.64\sym{***}&       -6.38\sym{**} \\
                    &      (1.05)         &      (1.33)         &      (2.62)         &      (2.87)         \\
Toposequence        &                     &         Yes         &         Yes         &         Yes         \\
Soil quality controls&                     &         Yes         &         Yes         &         Yes         \\
Subdivisional effects&                     &         Yes         &                     &                     \\
Plot size effects   &                     &         Yes         &                     &                     \\
                    &                     &                     &                     &                     \\
Mean dep. var       &        65.4         &        65.4         &        65.4         &        65.4         \\
Candidate Instruments&                     &                     &                     &                     \\
Instruments Selected&                     &                     &                     &                     \\
Unique Farmers      &                     &                     &                     &                     \\
Farmer-Crops        &        9290         &        9272         &        9272         &        9272         \\

	\bottomrule
	\multicolumn{5}{p{0.96\hsize}}{\footnotesize The table reports coefficients from regressions of agricultural output measures on well depth and controls. The data is from the main agricultural household survey and the observations are at the farmer-by-crop level. The dependent variable changes in each panel. In Panel A, the dependent variable is yield (quintals per Ha). In Panel B, the dependent variable is the value of output (INR per Ha), where the price for each crop is taken to be the median of the price reported at the SDO level. Well depth is the reported depth of a given farmer's well. Toposequence includes controls for elevation and slope. Subdivisional effects are dummy variables for each of the six sub-divisional offices of the distribution company from which farmers were sampled. Plot size effects are dummy variables indicating the plot size decile for each farmer-crop based on its plot area. Standard errors are clustered at the feeder level, the block within which farmers were randomly selected. The statistical significance of a coefficient at certain thresholds is indicated by  * $p < 0.10$, ** $p < 0.05$, *** $p < 0.01$.} \\
\end{tabular*}
\end{table}


\renewcommand{\tabcolsep}{2pt}
\begin{table}[htbp]\centering
\def\sym#1{\ifmmode^{#1}\else\(^{#1}\)\fi}
\caption{Instrumental variable estimates of farmer adaptation to water scarcity\label{tab:ivAdaptation}}
\begin{tabular*}{1\hsize}{@{\hskip\tabcolsep\extracolsep\fill}l*{5}{D{.}{.}{-1}}}
\toprule
&\multicolumn{1}{c}{IV-PDS}&\multicolumn{1}{c}{IV-PDS}&\multicolumn{1}{c}{IV-PDS}&\multicolumn{1}{c}{IV-PDS}&\multicolumn{1}{c}{IV-PDS}\\
\cmidrule(lr){2-2}\cmidrule(lr){3-3}\cmidrule(lr){4-4}\cmidrule(lr){5-5}\cmidrule(lr){6-6}
&\multicolumn{1}{c}{\shortstack{High-yielding \\ variety}}&\multicolumn{1}{c}{\shortstack{Parcel\\leveled}}&\multicolumn{1}{c}{\shortstack{Sprinkler \\ irrigated}}&\multicolumn{1}{c}{\shortstack{Furrow/Flood \\ irrigated}}&\multicolumn{1}{c}{\shortstack{Under \\ irrigated}}\\
&\multicolumn{1}{c}{(1)}&\multicolumn{1}{c}{(2)}&\multicolumn{1}{c}{(3)}&\multicolumn{1}{c}{(4)}&\multicolumn{1}{c}{(5)}\\
\midrule \\
Well depth (1 sd = 187 feet)&      -0.049         &      -0.080\sym{*}  &      -0.099\sym{**} &       0.051         &        0.12\sym{***}\\
                    &     (0.034)         &     (0.044)         &     (0.041)         &     (0.042)         &     (0.031)         \\
 
                    &                     &                     &                     &                     &                     \\
Mean dep. var       &        0.62         &        0.19         &        0.30         &        0.34         &        0.20         \\
Candidate Instruments&         419         &         419         &         419         &         419         &         419         \\
Instruments Selected&          11         &          10         &          11         &          11         &          10         \\
Unique Farmers      &        3998         &        3982         &        4006         &        4006         &        3982         \\
Farmer-Crops        &        8711         &        6857         &        9748         &        9748         &        6857         \\
\bottomrule
\multicolumn{6}{p{\hsize}}{\footnotesize This table shows instrumental variable regressions of potential margins of adaptation to water scarcity on farmer well depth. Each column presents estimates from a model with a different outcome variable, as shown in the column headers. The data is from the main agricultural household survey and the observations are at the farmer-by-crop level for all but the first column where the data is at the farmer-by-parcel level. All the model specifications control for the toposequence (elevation and slope), along with subdivisional and plot size effects, as defined in Table \ref{tab:ivProfitsDepth}. We use our preferred candidate instrument set which is labelled Main in Table \ref{tab:instruments} . Standard errors are clustered at the feeder, the primary sampling unit. The statistical significance of a coefficient at certain thresholds is indicated by  \sym{*} $ p < 0.10$, \sym{**} $ p < 0.05$, \sym{***} $ p < 0.01$.} \\
\end{tabular*}
\end{table}


\renewcommand{\tabcolsep}{8pt}
\begin{table}[htbp]\centering
\def\sym#1{\ifmmode^{#1}\else\(^{#1}\)\fi}
\caption{Robustness to choice of instrument sets}
\begin{tabular*}{1\hsize}{@{\hskip\tabcolsep\extracolsep\fill}l*{5}{D{.}{.}{-1}}}
\toprule
&\multicolumn{1}{c}{IV-2SLS}&\multicolumn{1}{c}{IV-PDS}&\multicolumn{1}{c}{IV-PDS}&\multicolumn{1}{c}{IV-PDS}&\multicolumn{1}{c}{IV-PDS}\\
\cmidrule(lr){2-2}\cmidrule(lr){3-3}\cmidrule(lr){4-4}\cmidrule(lr){5-5}\cmidrule(lr){6-6}
&\multicolumn{1}{c}{Fractures}&\multicolumn{1}{c}{Rock}&\multicolumn{1}{c}{Aquifers}&\multicolumn{1}{c}{Main}&\multicolumn{1}{c}{Large}\\
&\multicolumn{1}{c}{(1)}&\multicolumn{1}{c}{(2)}&\multicolumn{1}{c}{(3)}&\multicolumn{1}{c}{(4)}&\multicolumn{1}{c}{(5)}\\
\midrule \\

\addlinespace
\multicolumn{6}{c}{\emph{Panel A. Total Profit, reported ('000 INR per Ha)}} \\
\addlinespace
Well depth (1 sd = 187 feet)&       -1.23         &       -7.64\sym{***}&       -8.14\sym{***}&       -7.68\sym{***}&       -6.85\sym{**} \\
                    &      (17.2)         &      (2.60)         &      (2.60)         &      (2.51)         &      (2.72)         \\
                    &                     &                     &                     &                     &                     \\
Mean dep. var       &       -5.06         &       -5.06         &       -5.06         &       -5.06         &       -5.06         \\
Candidate Instruments&           3         &         130         &         153         &         419         &        1728         \\
Instruments Selected&                     &          11         &          12         &          16         &          17         \\
Unique Farmers      &        3999         &        3999         &        3999         &        3999         &        3999         \\
Farmer-Crops        &        8973         &        8973         &        8973         &        8973         &        8973         \\

\addlinespace
\addlinespace
\multicolumn{6}{c}{\emph{Panel B. Cash Profit ('000 INR per Ha)}} \\
\addlinespace
Well depth (1 sd = 187 feet)&       -47.2\sym{*}  &       -9.67\sym{*}  &       -15.8\sym{***}&       -9.62\sym{*}  &       -11.9\sym{**} \\
                    &      (26.0)         &      (5.35)         &      (5.97)         &      (4.95)         &      (5.73)         \\
                    &                     &                     &                     &                     &                     \\
Mean dep. var       &       -13.5         &       -13.5         &       -13.5         &       -13.5         &       -13.5         \\
Candidate Instruments&           3         &         130         &         153         &         419         &        1728         \\
Instruments Selected&                     &           5         &           5         &          10         &           7         \\
Unique Farmers      &        2121         &        2121         &        2121         &        2121         &        2121         \\
Farmer-Crops        &        3243         &        3243         &        3243         &        3243         &        3243         \\

\bottomrule
\multicolumn{6}{p{\hsize}}{\footnotesize This table shows instrumental variable regressions of different measures of agricultural output on farmer well depth.  The data is from the main agricultural household survey and the observations are at the farmer-by-crop level. The dependent variable changes by panel. In Panel A, the dependent variable is reported total profit (INR per Ha), in Panel B, it is cash profit. All model specifications control for the toposequence (elevation and slope), along with subdivisional and plot size effects, as defined in Table \ref{tab:ivProfitsDepth}. The set of candidate instruments changes by column; the definitions of different instrument sets used in the model specifications above can be found in Table \ref{tab:instruments}. Standard errors are clustered at the feeder level, the block within which farmers are randomly selected. The statistical significance of a coefficient at certain thresholds is indicated by  \sym{*} $ p < 0.10$, \sym{**} $ p < 0.05$, \sym{***} $ p < 0.01$.} \\
\end{tabular*}
\end{table}


\begin{table}[htbp]\centering
\def\sym#1{\ifmmode^{#1}\else\(^{#1}\)\fi}
\caption{Robustness to the choice of controls\label{ivRobustnessControlsProfit}}
\begin{tabular*}{1\hsize}{@{\hskip\tabcolsep\extracolsep\fill}l*{5}{D{.}{.}{-1}}}
\toprule
                    &\multicolumn{1}{c}{(1)}&\multicolumn{1}{c}{(2)}&\multicolumn{1}{c}{(3)}&\multicolumn{1}{c}{(4)}&\multicolumn{1}{c}{(5)}\\
                    &\multicolumn{1}{c}{IV-PDS}&\multicolumn{1}{c}{IV-PDS}&\multicolumn{1}{c}{IV-PDS}&\multicolumn{1}{c}{IV-PDS}&\multicolumn{1}{c}{IV-PDS}\\
\midrule

\addlinespace
\multicolumn{6}{c}{\emph{Panel A. Total Profit (INR per Ha)}} \\
\addlinespace
Well depth (1 sd = 187 feet)&       -9.66\sym{***}&       -8.91\sym{***}&       -8.83\sym{***}&       -8.44\sym{***}&       -6.03\sym{**} \\
                    &      (2.61)         &      (2.75)         &      (2.63)         &      (2.41)         &      (2.57)         \\
Subdivisional effects&                     &                     &                     &                     &                     \\
Plot size effects   &                     &                     &                     &                     &                     \\
Toposequence        &                     &                     &         Yes         &         Yes         &         Yes         \\
Soil quality controls&                     &                     &                     &         Yes         &         Yes         \\
Temperature         &                     &                     &                     &                     &         Yes         \\
                    &                     &                     &                     &                     &                     \\
Mean dep. var       &       -5.06         &       -5.06         &       -5.06         &       -5.06         &       -5.06         \\
Candidate Instruments&                     &                     &                     &                     &                     \\
Instruments Selected&                     &                     &                     &                     &                     \\
Unique Farmers      &                     &                     &                     &                     &                     \\
Farmer-Crops        &        8991         &        8991         &        8973         &        8973         &        8973         \\

\addlinespace
\multicolumn{6}{c}{\emph{Panel B. Cash Profit (INR per Ha)}} \\
\addlinespace
Well depth (1 sd = 187 feet)&       -11.5\sym{**} &       -9.91\sym{*}  &       -9.28\sym{*}  &       -9.62\sym{*}  &       -10.2\sym{**} \\
                    &      (5.06)         &      (5.08)         &      (4.77)         &      (4.95)         &      (4.84)         \\
Subdivisional effects&         Yes         &         Yes         &         Yes         &         Yes         &         Yes         \\
Plot size effects   &                     &         Yes         &         Yes         &         Yes         &         Yes         \\
Toposequence        &                     &                     &         Yes         &         Yes         &         Yes         \\
Soil quality controls&                     &                     &                     &         Yes         &         Yes         \\
Temperature         &                     &                     &                     &                     &         Yes         \\
                    &                     &                     &                     &                     &                     \\
Mean dep. var       &       -13.5         &       -13.5         &       -13.5         &       -13.5         &       -13.5         \\
Candidate Instruments&         419         &         419         &         419         &         419         &         419         \\
Instruments Selected&          10         &           9         &          12         &          10         &           7         \\
Unique Farmers      &        2127         &        2127         &        2121         &        2121         &        2121         \\
Farmer-Crops        &        3253         &        3253         &        3243         &        3243         &        3243         \\

\bottomrule
\multicolumn{6}{p{1.0\hsize}}{\footnotesize This table shows instrumental variable regressions of different measures of agricultural output on farmer well depth.  The data is from the main agricultural household survey and the observations are at the farmer-by-crop level. The dependent variable changes by panel. In Panel A, the dependent variable is reported cash profit (INR per Ha), in Panel B, it is total profit which is inclusive of the value of the farmer's own consumption (INR per Ha). All models use the main instrument set as described in Table \ref{tab:instruments}. The set of controls included changes by column; for example, the first column only includes subdivisional effects whereas the last column includes all five sets of controls considered. Standard errors are clustered at the feeder level, the primary sampling unit. The statistical significance of a coefficient at certain thresholds is indicated by  \sym{*} $ p < 0.10$, \sym{**} $ p < 0.05$, \sym{***} $ p < 0.01$.} 
\end{tabular*}
\end{table}


\clearpage

\begin{figure}[htbp]
	\centering\caption{Depths of wells dug by year\label{fig:wellsDugByYear}}
	\includegraphics[scale=1]{fig_wells_dug_by_year}
	\begin{tabular*}{1.0\hsize}{c}
		\multicolumn{1}{p{1.0\hsize}}{\footnotesize This figure shows the distribution of depths of wells dug by farmers in our sample between the years 1990 and 2016.}\\
	\end{tabular*}
\end{figure}

\begin{table}[!ht]
	\centering
	\caption{Parameters used in the dynamic model \label{tab:dynamicParameters}} 
	\begin{tabular}{p{2 cm}p{2 cm}p{6 cm}}
		\toprule
		 Parameter & Value & Source \\
		\midrule
		\multicolumn{3}{c}{\emph{Primitives}} \\
 		\addlinespace 
		 $\alpha_W$ & 0.18 & Main model \\
		 $\Omega$ & 13.00 & Main model \\
		\addlinespace
 		\multicolumn{3}{c}{\emph{Exogenous variables}} \\
 		\addlinespace 
		 $p_E$ & INR 0.9  & Rajasthan policy \\
		 $c_E$ & INR 6.2  & Rajasthan policy \\
		 $\overline{H}$ & 6 hours & Rajasthan policy \\
		\bottomrule
		\multicolumn{3}{p{0.65\hsize}}{\footnotesize This table reports the inputs to our model that are homogenous across all SDOs.The \emph{primitives} are unobserved structural parameters assumed to be policy invariant.These include $\alpha_W$, which defines the concavity of the production function,and $\Omega$ which is total factor productivity.The \emph{exogenous variables} are unmodeled policy choices which include the nominal price of one kilowatt-hour of electricity,the marginal cost of producing one kilowatt-hour of electricity, and the power ration in hours per day.}
	\end{tabular}
\end{table}

\begin{table}[!ht]
	\centering
		\caption{Estimates of $\lambda_W$ for alternate parameter values \label{tab:oppCostParamliter}} 
	\begin{tabular}{p{1 cm}p{2 cm}p{2 cm}p{2 cm}p{2 cm}p{2 cm}}
		\toprule
		$\beta\backslash\alpha_W$&0.12&0.15&0.18&0.21&0.24\\
		\midrule
		\addlinespace
		\multirow{2}{1 cm}{0.95}&1.94&3.00&4.46&6.44&9.13\\
		&(0.09)&(0.13)&(0.20)&(0.28)&(0.40)\\
		\addlinespace
		\addlinespace
		\multirow{2}{1 cm}{0.90}&1.42&2.19&3.27&4.72&6.69\\
		&(0.09)&(0.14)&(0.21)&(0.30)&(0.42)\\
		\addlinespace
		\addlinespace
		\multirow{2}{1 cm}{0.75}&0.72&1.12&1.67&2.41&3.42\\
		&(0.07)&(0.10)&(0.15)&(0.22)&(0.31)\\
		\bottomrule
		\multicolumn{6}{p{0.90\hsize}}{\footnotesize This table reports the opportunity cost of water for different values of the output elasticity of water $\alpha_W $ and the discount rate $\beta $. The units of $\lambda_W$ are INR per liter. Bootstrapped standard errors in parentheses account for estimation error in the groundwater law of motion.}
	\end{tabular}
\end{table}

\end{document}